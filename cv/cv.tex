%%%%%%%%%%%%%%%%%%%%%%%%%%%%%%%%%%%%%%%%%
% "ModernCV" CV and Cover Letter
% LaTeX Template
% Version 1.11 (19/6/14)
%
% This template has been downloaded from:
% http://www.LaTeXTemplates.com
%
% Original author:
% Xavier Danaux (xdanaux@gmail.com)
%
% License:
% CC BY-NC-SA 3.0 (http://creativecommons.org/licenses/by-nc-sa/3.0/)
%
% Important note:
% This template requires the moderncv.cls and .sty files to be in the same
% directory as this .tex file. These files provide the resume style and themes
% used for structuring the document.
%
%%%%%%%%%%%%%%%%%%%%%%%%%%%%%%%%%%%%%%%%%

%----------------------------------------------------------------------------------------
%	PACKAGES AND OTHER DOCUMENT CONFIGURATIONS
%----------------------------------------------------------------------------------------

\documentclass[11pt,a4paper,sans]{moderncv} % Font sizes: 10, 11, or 12; paper sizes: a4paper, letterpaper, a5paper, legalpaper, executivepaper or landscape; font families: sans or roman
\usepackage[utf8]{inputenc}
\usepackage[polish]{babel}

\usepackage{etaremune}

\moderncvstyle{classic} % CV theme - options include: 'casual' (default), 'classic', 'oldstyle' and 'banking'
\moderncvcolor{blue} % CV color - options include: 'blue' (default), 'orange', 'green', 'red', 'purple', 'grey' and 'black'

\usepackage{lipsum} % Used for inserting dummy 'Lorem ipsum' text into the template

\usepackage[scale=0.75]{geometry} % Reduce document margins


%\setlength{\hintscolumnwidth}{3cm} % Uncomment to change the width of the dates column
%\setlength{\makecvtitlenamewidth}{10cm} % For the 'classic' style, uncomment to adjust the width of the space allocated to your name

%----------------------------------------------------------------------------------------
%	NAME AND CONTACT INFORMATION SECTION
%----------------------------------------------------------------------------------------

\firstname{Łukasz} % Your first name
\familyname{Machura} % Your last name

% All information in this block is optional, comment out any lines you don't need
\title{Curriculum Vitae}
%\address{ul. Turniejowa}{3/12 Będzin}
%\mobile{(+48) 664-311-717}
\email{lukasz.machura@us.edu.pl}
\homepage{https://lukaszmachura.github.io/} {lukaszmachura.github.io} % The first argument is the url for the clickable link, the second argument is the url displayed in the template - this allows special characters to be displayed such as the tilde in this example
%\extrainfo{additional information}
% \photo[70pt][0pt]{lm} % The first bracket is the picture height, the second is the thickness of the frame around the picture (0pt for no frame)
%\quote{"A witty and playful quotation" - John Smith}

%----------------------------------------------------------------------------------------

\begin{document}

\makecvtitle % Print the CV title

%-----------------------------------------------------------https://www.overleaf.com/project/5fb18c028fd1d3e28834aca3-----------------------------
%	EDUCATION SECTION
%----------------------------------------------------------------------------------------

\section{Edukacja/Praca Naukowa}

%\cventry{2002--2005}{IV LO im. Hanki Sawickiej, Kielce}{}{}{}{}
\cventry{2016}{Habilitacja, dziedzina nauk fizycznych, Uniwersytet Śląski w Katowicach}{}{}{}{}

\cventry{2006}{Doktorat, dziedzina nauk fizycznych, fizyka statystyczna, Uniwersytet w Augsburgu, Niemcy}{}{}{}{}

\cventry{2003--2006}{Studia doktoranckie na Uniwersytecie w Augsburgu, 3 letnie stypendium w programie Graduiertenkolleg 283}{}{}{}{}

\cventry{2002--2003}{Studia doktoranckie na Uniwersytecie w Augsburgu, roczne stypendium w programie DAAD}{}{}{}{}

\cventry{2001--2002}{Studia doktoranckie na Uniwersytecie Śląskim w Katowicach}{}{}{}{}

\cventry{1996--2001}{Studia magisterskie z Fizyki na Uniwersytecie Śląskim w Katowicach, specjalność fizyka teoretyczna}{}{}{}{}

% Arguments not required can be left empty



% \cventry{2014--2019}{Studia doktoranckie, kierunek: Fizyka, Uniwersytet Śląski}{}{}{}{}
% \cventry{2020}{Uzyskanie stopnia doktora nauk fizycznych}{}{}{}{}


%\section{Masters Thesis}

%\cvitem{Title}{\emph{Technologies and characterization of ferroelectric polymers for biomedical sensors}}
%\cvitem{Supervisors}{Professor Antonino Fiorillo}
%\cvitem{Description}{This thesis is based on the implementation of a temperature sensor.}

%----------------------------------------------------------------------------------------
%	WORK EXPERIENCE SECTION
%----------------------------------------------------------------------------------------


\section{Zainteresowania naukowe}
% \renewcommand{\listitemsymbol}{-~} % Changes the symbol used for lists
\cventry{-}{Nieliniowe metody analizy sygnałów elektrofizjologicznych}{}{}{}{}
\cventry{-}{Metody statystyczne i analiza danych}{}{}{}{}
\cventry{-}{Modelowanie statystyczne oparte na danych}{}{}{}{}
\cventry{-}{Fizyka statystyczna nierównowagowych układów otwartych}{}{}{}{}
\cventry{-}{Analiza i modelowanie komputerowe układów złożonych}{}{}{}{}
\cventry{-}{Metody numeryczne, programowanie rozproszone}{}{}{}{}


\section{Udział w wybranych projektach}
\cventry{-}{CC Infinite. Computing Competences. Innovative learning approach for non-IT students, kierownik}{}{}{}{}
\cventry{-}{Wpływ zmęczenia na aktywność mięśniową kończyn górnych z wykorzystaniem powierzchniowej elektromiografii kinezjologicznej}{}{}{}{}
\cventry{-}{pRobot. Nauka programowania z wykorzystaniem robotów}{}{}{}{}
\cventry{-}{Od Algorytmu do Zawodowca}{}{}{}{}
\cventry{-}{DAAD, Dissipative transport and ordering in complex systems}{}{}{}{}
\cventry{-}{DAAD, Stochastic complexity: Rate theory, nonexponential decay of fluctuations and stochastic networks}{}{}{}{}
\cventry{-}{European Science Foundation „Stochastic Dynamics: fundamentals and applications (STOCHDYN)}{}{}{}{}
\cventry{-}{Program BIOCenter projektu K4@A4}{}{}{}{}



\section{Inżynieria komputerowa}
\cventry{-}{Biegłe programowanie w językach Python, C, CUDA, bash}{}{}{}{} \cventry{-}{Znajomość języków C++, Java, PHP, JavaScript}{}{}{}{}
\cventry{-}{Języki opisu treści: \LaTeX, HTML+CSS, Markdown, rsText}{}{}{}{}

\cventry{-}{Administracja serwerami Linux}{}{}{}{}

% \newpage

% \begin{center}
%     \Large{Dorobek naukowy}
% \end{center}

\section{Lista publikacji naukowych}

\begin{etaremune}  %add \usepackage{etaremune}
  \item Multifractal Properties of BK Channels’ Currents in Human Glioblastoma Cells, A. Wawrzkiewicz-Jałowiecka, P. Trybek, B. Dworakowska and L. Machura, J. Phys. Chem. B 124,  12  2382-2391  (2020)

\item Differences in Gating Dynamics of BK Channels in Cellular and Mitochondrial Membranes from Human Glioblastoma Cells Unraveled by Short- and Long-Range Correlations Analysis, A. Wawrzkiewicz-Jałowiecka, P. Trybek, P. Borys, B. Dworakowska, L. Machura and P. Bednarczyk, Cells 9,  2305 (2020)

\item The distribution of information for sEMG signals in the rectal cancer treatment process, P. Trybek, M. Nowakowski, J. Salowka and L. Machura, BioSystems 176,  13 (2019)

\item Upper extremity surface electromyography signal changes after laparoscopic training, M. Nowakowski, P. Trybek, M. Rubinkiewicz, T. Cegielny, M. Romaniszyn, M. Pędziwiatr, L. Machura, Videosurgery Miniinv,  (2018)  doi:10.5114/wiitm.2018.78744

\item Single measurement detection of individual cell ionic oscillations using an n-type semiconductor – electrolyte interface, M. Pietruszka, M. Olszewska, L. Machura and E. Rowinski, Sci. Rep. 8,  7875 (2018)

\item Inter-pulse intervals of external anal sphincter surface EMG signals recorded from colorectal cancer patients, L. Machura, P. Trybek, J. Salowka and M. Nowakowski, submitted (2018)
,
\item Multifractal characteristics of external anal sphincter based on sEMG signals, P. Trybek, M. Nowakowski and L. Machura, Med Eng Phys 55,  9 (2018)

\item Mechanosensitivity of the BK channels in human glioblastoma cells - kinetics and dynamical complexity, A. Wawrzkiewicz-Jalowiecka, P. Trybek, L. Machura, B. Dworakowska, Z. J. Grzywna, J Membrane Biol,  1-13 (2018)

\item  Sample entropy of sEMG signals at different stages of rectal cancer treatment, P. Trybek, M. Nowakowski, J. Salowka, J. Spiechowicz, L. Machura, Entropy 20,  863 (2018)

\item Sensitivity and specificity of multichannel surface electromyography in diagnosing fecal incontinence, M. Nowakowski, K. A. Tomaszewski, P. Trybek, L. Machura, R. M. Herman, Folia Med Cracov. LVII 1,  29 (2017)

\item Correlation based analysis of sEMG signals during complex muscle activity. Feasibility study of new methodology., M. Nowakowski, P. Trybek and L. Machura, Folia Med Cracov. LVII 2,  41 (2017)

\item Efficiency of transport in periodic potentials: dichotomous noise contra deterministic force, J. Spiechowicz,  J. Luczka and L. Machura, J. Stat. Mech.,  054038 (2016)

\item Evaluation of the training objectives with surface electromyography, P. Trybek, M. Nowakowski and L. Machura, Bio-Algorithms and Med-Systems 12,  25 (2016)

\item Persistent currents in metallic rings containing a quantum dot, L. Machura and J. Luczka, Phys. Lett. A 379,  1654 (2015)

\item GPU accelerated Monte Carlo simulation of Brownian motors dynamics with CUDA, J. Spiechowicz, M. Kostur and L. Machura, Comput. Phys. Commun. 191,  140 (2015)

\item Directed transport in coupled noisy Josephson junctions controlled via ac signals, L. Machura, J. Spiechowicz and J. Luczka, Phys. Scr. 151,  014021 (2012)

\item Two coupled Josephson junctions: dc voltage controlled by biharmonic current, L. Machura, J. Spiechowicz, M. Kostur and J. Łuczka, J. Phys. Condens. Matter 24,  085702 (2012)

\item Control of transport characteristics in two coupled Josephson junctions, J. Spiechowicz, L. Machura, M. Kostur and J. Łuczka, Acta Phys. Polon. B 43,  1203 (2012)

\item Transport driven by biharmonic forces: impact of correlated thermal noise, Ł. Machura and J. Łuczka, Phys. Rev. E 82,  031133 (2010)

\item Negative conductances of Josephson junctions: Voltage fluctuations and energetics, Ł. Machura, M. Kostur, P. Talkner, P. Hänggi, and J. Łuczka, Phys. E 42,  590 (2010)

\item Current-flux characteristics in mesoscopic nonsuperconducting rings, Ł. Machura, S. Rogoziński, and J. Łuczka, Fast Track J. Phys.: Condens. Matter 22,  422201 (2010)

\item Current characteristics of mesoscopic rings in quantum Smoluchowski regime, S. Rogoziński, Ł. Machura and J. Łuczka, Eur. Phys. J. - Special Topics 187,  5 (2010)

\item Inertial Brownian motors driven by biharmonic signals, Ł. Machura, M. Kostur, and J. Łuczka, Chemical Physics 375,  445 (2010)

\item Transmission of magnetic signals in noisy mesorings, L. Machura, J. Dajka and J. Łuczka, J. Stat. Mech.,  P01030 (2009)

\item Anomalous transport in biased ac-driven Josephson junctions: Negative conductances, M. Kostur, L. Machura, P. Talkner, P. Hänggi, and J. Łuczka, Phys. Rev. B 77,  104509 (2008)

\item Mesoscopic rings: multi-states induced by quantum thermal fluctuations, J. Dajka, L. Machura, S. Rogoziński and J. Łuczka, Materials Science Poland 26,  871 (2008)

\item Transport Characteristics of Molecular Motors, L. Machura, M. Kostur and J. Łuczka, biosystems 05,  033 (2008)

\item Negative Conductance in Driven Josephson Junctions, M. Kostur, Ł. Machura, J. Łuczka, P. Talkner, P. Hänggi, Acta Physica Polonica B 39,  1115 (2008)

\item Absolute negative mobility induced by thermal equilibrium fluctuations, L. Machura, M. Kostur, P. Talkner, J. Luczka and P. Hänggi, Phys. Rev. Lett. 98,  040601 (2007)

\item Magnetic flux in mesoscopic rings: Quantum Smoluchowski regime, J. Dajka, Ł. Machura, S. Rogoziński, and J. Łuczka, Phys. Rev. B 76,  045337 (2007)

\item Flux-biased mesoscopic rings, J. Dajka, L. Machura, S. Rogoziński and J. Łuczka, phys. stat. sol (b) 244,  2432 (2007)

\item Magnetic flux in mesoscopic rings: capacity, inertia and kinetics, J. Dajka, S. Rogoziński, Ł. Machura, and J. Łuczka, Acta Physica Polonica B 38,  1737 (2007)

\item Transport of Forced Quantum Motors in the Strong Friction Limit, L. Machura, J. Luczka, P. Talkner and P. Hänggi, Acta Physica Polonica B 38,  1855 (2007)

\item Frequency Windows of Absolute Negative Conductance in Josephson Junctions, L. Machura, M. Kostur, P. Talkner, P. Hänggi, and J. Luczka, AIP Conference Proceedings 922,  455 (2007)

\item Quantum diffusion in biased washboard potentials: strong friction limit, L. Machura, M. Kostur, P. Talkner, J. Luczka and P. Hänggi, Phys. Rev. E 73,  031105 (2006)

\item Forcing inertial Brownian motors: efficiency and negative differential mobility, M. Kostur, L. Machura, P. Hänggi, J. Łuczka and P. Talkner, Physica A 371,  20 (2006)

\item Addendum and Erratum: Optimal strategy for controlling transport in inertial Brownian motion, L. Machura, M. Kostur, F. Marchesoni, P. Talkner, P. Hänggi and J. Luczka, J. Phys.: Condens. Matter 18,  4111-4112 (2006)

\item Statistics of transition times, phase diffusion and synchronization in periodically driven bistable systems, Peter Talkner, Lukasz Machura, Michael Schindler, Peter Hänggi and Jerzy Luczka, New J. Phys. 7,  14 (2005)

\item Optimal strategy for controlling transport in inertial Brownian motion, L. Machura, M. Kostur, F. Marchesoni, P. Talkner, P. Hänggi and J. Luczka, J. Phys.: Condens. Matter 17,  S3741-S3752 (2005)

\item Brownian motors: current fluctuations and rectification effciency, L. Machura, M. Kostur, P. Talkner, J. Luczka, F. Marchesoni and P. Hänggi, Phys. Rev. E 70,  061105 (2004)

\item Consistent Description Of Quantum Brownian Motors Operating At Strong Friction, L. Machura, M. Kostur, P. Hänggi, P. Talkner and J. Luczka, Phys. Rev. E 70,  031107 (2004)

\end{etaremune}


% \section{Udział w międzynarodowych konferencjach naukowych}
% \begin{itemize}
% \item School of information transmission in Biological System, Bedlewo, 5-9 March 2018,
% \textit{Evolution of Information within sEMG signals in the process of bowel cancer treatment (wykład)}
% \item Cardiology Meets Physics and Mathematics-Electrophysiology: Basic and Applied, 1-3
% March 2018, Zakopane, \textit{Entropy-based approach to classification the surface
% electromyographic signals (wykład)}
% \item Cardiology Meets Physics and Mathematics: 6-9 March 2019, Zakopane, \textit{Multifractal
% characteristics of BK channels in human glioblastoma cells (wykład)}.
% \item 31-th Smoluchowski Symposium on Statistical Physics, Zakopane 3-9.09.2018: \textit{Multifractals characteristics of BK channels in human glioblastoma cells}.
% \item 42 International Conference of Theoretical Physics, Correlation and Coherence
% at Different Scales: \textit{Multifractals characteristics of BK channels in human
% glioblastoma cells}, Ustroń, 10-14.09.2018.
% \item 30-th Marian Smoluchowski Symposium, 3-8 September 2017, Kraków, \textit{Entropy-based analysis of the surface electromyography signals}.
% \item Intel Code Modernization Workshop, 29-30 May 2017, Poznan, \textit{ Nonlinear analysis of
% electrophysiological signals}.
% \item 8 International Workshop on Biosignal Interpretation, Osaka, Japan, 1-3 November
% 2016, \textit{Multifractal characteristics of external anal sphincter based on sEMG signal}.

% \item 29-th Marian Smoluchowski Symposium on Statistical Physics, Zakopane,
% 12-16 September 2016, \textit{Multifractal analysis based on empirical mode decomposition
% applied to electrophysiological signals}.
% \item 40--th International Conference on Theoretical Physics, 5-9 September 2016,
% Ustron, \textit{Electrophysiological time series analysis using a modified algorithm of
% detrended fluctuation analysis based on empirical mode decomposition}.
% \item 28th Marian Smoluchowski Symposium on Statistical Physics, 14-17 September
% 2015, Kraków, \textit{Application of Multifractal Detrended Fluctuation Analysis in
% biomedical signals}.
% \item International Conference MCSB, Cybernetic Modeling of Biological System, 14-15
% May, Kraków 2015, \textit{The multifractal analysis of the kinesiological surface electromyography signal}.
% \item 27 Marian Smoluchowski Symposium on Statistical Physics. Fundamentals, soft
% matter and biocomplexity (September 22-26, Zakopane 2014), \textit{Multifractal analysis
% of sEMG signal of the complex muscle activity}.
% \item XXXVIII International Conference of Theoretical Physics Correlations and Coherence
% at Different scales, 05-10 September 2014, Ustroń, Poland, \textit{Multifractal properties of
% the electrophysiological time series}.

% \end{itemize}




%----------------------------------------------------------------------------------------
%	AWARDS SECTION
%----------------------------------------------------------------------------------------





%----------------------------------------------------------------------------------------
%	COMPUTER SKILLS SECTION
%----------------------------------------------------------------------------------------


%----------------------------------------------------------------------------------------
%	COMMUNICATION SKILLS SECTION
%----------------------------------------------------------------------------------------

%\section{Communication Skills}

%\cvitem{2010}{Oral Presentation at the California Business Conference}
%\cvitem{2009}{Poster at the Annual Business Conference in Oregon}

%----------------------------------------------------------------------------------------
%	LANGUAGES SECTION
%----------------------------------------------------------------------------------------
% \vspace{-0.5cm}
% \section{Złożone projekty NCN}

% \cventry{-}{Od nowicjusza do eksperta w dziedzinie laparoskopii. Złożoność, koherencja i teoria informacji w aspekcie procesów kognitywnych związanych z koordynacją wzrokowo-motoryczną (konkurs preludium)}{}{}{}{}
% %\cvitemwithcomment{Dutch}{Basic}{Basic words and phrases only}

%----------------------------------------------------------------------------------------
%	INTERESTS SECTION
%----------------------------------------------------------------------------------------


%\cvlistitem{teoria informacji w analizie sygnałów}
%\cvlistitem{ dynamika kanałów jonowych}


%----------------------------------------------------------------------------------------
%	COVER LETTER
%----------------------------------------------------------------------------------------

% To remove the cover letter, comment out this entire block

%\clearpage

%\recipient{HR Department}{Corporation\\123 Pleasant Lane\\12345 City, State} % Letter recipient
%\date{\today} % Letter date
%\opening{Dear Sir or Madam,} % Opening greeting
%\closing{Sincerely yours,} % Closing phrase
%\enclosure[Attached]{curriculum vit\ae{}} % List of enclosed documents

%\makelettertitle % Print letter title

%\lipsum[1-3] % Dummy text

%\makeletterclosing % Print letter signature

%----------------------------------------------------------------------------------------

\end{document}
